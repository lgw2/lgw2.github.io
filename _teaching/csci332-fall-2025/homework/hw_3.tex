% ---------
% --------
%!TEX TS-program = pdflatex
%!TEX encoding = UTF-8 Unicode
%!TEX spellcheck = English (Aspell)

\documentclass[11pt]{article}
\usepackage{jeffe,handout,graphicx}
\usepackage{fourier-orns}
\usepackage{mathtools}
\usepackage[normalem]{ulem} % either use this (simple) or
\usepackage{tikz}

\def\Cdot{\mathbin{\text{\normalfont \textbullet}}}
\def\Sym#1{\texttt{\color{BrickRed}#1}}
\def\BlueSym#1{\textbf{\texttt{\color{RoyalBlue}#1}}}
\allowdisplaybreaks

\begin{document}

\begin{center}
\Large\textbf{CSCI 332, Fall 2025}%
\\
\LARGE\textbf{Homework 3}%
\\[0.5ex]
\large Due Monday, September 15 Anywhere on Earth (6am Tuesday)
\end{center}

\bigskip
\hrule
\bigskip

\subsection*{Submission Requirements}
\begin{itemize}
    \item Type or clearly hand-write your solutions into a PDF format so that they are legible and professional. Submit your PDF on Gradescope. 
    \item Do not submit your first draft. Type or clearly re-write your solutions for your final submission. If your submission is not legible, we will ask you to resubmit.
    \item Use Gradescope to assign problems to the correct page(s) in your solution. If you do not do this correctly, we will ask you to resubmit.
\end{itemize}

\subsection*{Academic Integrity}

Remember, you may access \EMPH{any} resource in preparing your solution to the homework. However, you \EMPH{must}
\begin{itemize}
    \item write your solutions in your own words, and
    \item credit every resource you use (for example: ``Bob Smith helped me on
    this problem. He took this course at UM in Fall 2020''; ``I found a solution
    to a problem similar to this one in the lecture notes for a different
    course, found at this link: www.profzeno.com/agreatclass/lecture10''; ``I
    asked ChatGPT how to solve  part (c)"; ``I put my solution for part (c) into
    ChatGPT to check that it was correct and it caught a missing case.'') If you
    use the provided LaTeX template, you can use the \texttt{sources}
    environment for this. Ask if you need help!
\end{itemize}



\newpage
%----------------------------------------------------------------------

\headers{CSCI 332}{Homework 3 (due September 15)}{Fall 2025}

\begin{enumerate}

\item Prove that $\log_a n = \Theta(\log_b n)$ for any positive constants $a, b
> 1$ by proving each of the following. You must use the definition of big-O and
big-Omega in your proofs; that is, you must give a $c$ and an $n_0$. (Hint: Use the change of base formula for logarithms,
which states that $\log_a n = \frac{\log_b n}{\log_b a}$.)

    \begin{enumerate}[(a)]
        \item $log_a n = O(log_b n)$
        \item $log_a n = \Omega(log_b n)$
    \end{enumerate} 

    
    \item (1 point) What resources did you use for this assignment? (If you only used the textbook, lecture notes, and office hours, you can say "none".)
    \end{enumerate}
    %%%%%%%%%%%%%%%%%%%%
    \end{document}
