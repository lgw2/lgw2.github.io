% ---------
% --------
%!TEX TS-program = pdflatex
%!TEX encoding = UTF-8 Unicode
%!TEX spellcheck = English (Aspell)

\documentclass[11pt]{article}
\usepackage{jeffe,handout,graphicx}
\usepackage{fourier-orns}
\usepackage{mathtools}
\usepackage[normalem]{ulem} % either use this (simple) or
\usepackage{tikz}

\def\Cdot{\mathbin{\text{\normalfont \textbullet}}}
\def\Sym#1{\texttt{\color{BrickRed}#1}}
\def\BlueSym#1{\textbf{\texttt{\color{RoyalBlue}#1}}}
\allowdisplaybreaks

\begin{document}

\begin{center}
\Large\textbf{CSCI 332, Fall 2025}%
\\
\LARGE\textbf{Homework 8}%
\\[0.5ex]
\large Due Monday, November 3 Anywhere on Earth (6am Tuesday)
\end{center}

\bigskip
\hrule
\bigskip

\subsection*{Submission Requirements}
\begin{itemize}
    \item Type or clearly hand-write your solutions into a PDF format so that they are legible and professional. Submit your PDF on Gradescope. 
    \item Do not submit your first draft. Type or clearly re-write your solutions for your final submission. If your submission is not legible, we will ask you to resubmit.
    \item Use Gradescope to assign problems to the correct page(s) in your solution. If you do not do this correctly, we will ask you to resubmit.
\end{itemize}

\subsection*{Academic Integrity}

Remember, you may access \EMPH{any} resource in preparing your solution to the homework. However, you \EMPH{must}
\begin{itemize}
    \item write your solutions in your own words, and
    \item credit every resource you use (for example: ``Bob Smith helped me on
    this problem. He took this course at UM in Fall 2020''; ``I found a solution
    to a problem similar to this one in the lecture notes for a different
    course, found at this link: www.profzeno.com/agreatclass/lecture10''; ``I
    asked ChatGPT how to solve  part (c)"; ``I put my solution for part (c) into
    ChatGPT to check that it was correct and it caught a missing case.'') If you
    use the provided LaTeX template, you can use the \texttt{sources}
    environment for this. Ask if you need help!
\end{itemize}



\newpage
%----------------------------------------------------------------------

\headers{CSCI 332}{Homework 8 (due November 3)}{Fall 2025}


This written homework will extend on the PrairieLearn assignment's final problem:

You are given a sequence of digits $A[1..n]$ where each digit is between 1 and 9
(inclusive). You are asked to insert $+$ signs in between the digits to partition them
into terms which will be added together. The length of each term in the summation
must be either 1 or 2 digits.

What is the maximum sum that can be achieved under these constraints?

For example, if $A=[1,9,1,2]$, the maximum sum is 1 + 91 + 2 = 94.

\begin{enumerate}

\item (0.5 point) The PrairieLearn assignment asked you to write a viable subproblem definition
that can be used to solve this problem. Copy your subproblem definition here.

\item (0.5 point) Write a recursive definition for your subproblem. Make sure to include base case(s) as needed!

\item (0.5 point) Describe how you will memoize your recursive solution. What data structure will you use and
in what order will you fill it?

\item (1.5 points) Write pseudocode for a dynamic programming algorithm that uses your
subproblem definition to solve the problem.

\item (1.5 points) Implement your pseudocode in the programming language of your choice and use your implementation to
verify that your algorithm is correct. This means that you should run your code on a few examples.
Copy your code and the output on your examples here.

\item (0.5 point) What outside resources did you use to help with this assignment?

\end{enumerate}
    %%%%%%%%%%%%%%%%%%%%
    \end{document}
