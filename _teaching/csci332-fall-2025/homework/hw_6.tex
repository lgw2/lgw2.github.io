% ---------
% --------
%!TEX TS-program = pdflatex
%!TEX encoding = UTF-8 Unicode
%!TEX spellcheck = English (Aspell)

\documentclass[11pt]{article}
\usepackage{jeffe,handout,graphicx}
\usepackage{fourier-orns}
\usepackage{mathtools}
\usepackage[normalem]{ulem} % either use this (simple) or
\usepackage{tikz}

\def\Cdot{\mathbin{\text{\normalfont \textbullet}}}
\def\Sym#1{\texttt{\color{BrickRed}#1}}
\def\BlueSym#1{\textbf{\texttt{\color{RoyalBlue}#1}}}
\allowdisplaybreaks

\begin{document}

\begin{center}
\Large\textbf{CSCI 332, Fall 2025}%
\\
\LARGE\textbf{Homework 6}%
\\[0.5ex]
\large Due Monday, October 13 Anywhere on Earth (6am Tuesday)
\end{center}

\bigskip
\hrule
\bigskip

\subsection*{Submission Requirements}
\begin{itemize}
    \item Type or clearly hand-write your solutions into a PDF format so that they are legible and professional. Submit your PDF on Gradescope. 
    \item Do not submit your first draft. Type or clearly re-write your solutions for your final submission. If your submission is not legible, we will ask you to resubmit.
    \item Use Gradescope to assign problems to the correct page(s) in your solution. If you do not do this correctly, we will ask you to resubmit.
\end{itemize}

\subsection*{Academic Integrity}

Remember, you may access \EMPH{any} resource in preparing your solution to the homework. However, you \EMPH{must}
\begin{itemize}
    \item write your solutions in your own words, and
    \item credit every resource you use (for example: ``Bob Smith helped me on
    this problem. He took this course at UM in Fall 2020''; ``I found a solution
    to a problem similar to this one in the lecture notes for a different
    course, found at this link: www.profzeno.com/agreatclass/lecture10''; ``I
    asked ChatGPT how to solve  part (c)"; ``I put my solution for part (c) into
    ChatGPT to check that it was correct and it caught a missing case.'') If you
    use the provided LaTeX template, you can use the \texttt{sources}
    environment for this. Ask if you need help!
\end{itemize}



\newpage
%----------------------------------------------------------------------

\headers{CSCI 332}{Homework 6 (due October 13)}{Fall 2025}

\begin{enumerate}


\item 
Every fall, your family harvests apples at the old family orchard. You have many varied sizes of bins that you harvest the apples into, and a large fleet of identical carts that can all take a total weight of $W$. Your family has always done it exactly this way:
\begin{itemize}
    \item Park a cart at the orchard.
    \item As bins of apples are finished, pack them onto the cart in exactly the order they arrive.
    \item Once a bin arrives that would put the cart over its maximum weight, send the cart off and start loading the next cart.
\end{itemize}

According to this system, we could index the bins with the variable $i$ and
write the weight of the $i$\textsuperscript{th} bin $w_i$. Because your family
members pick apples at different speeds and fill the bins to different levels,
the $w_i$ values are varied and unpredictable, and because the apple orchard
produces at different levels every year, the total number of bins that will be
loaded varies year to year as well.

Since your family knows that you are taking advanced algorithms, they wonder if
you could help them decide whether they might be using too many carts. Maybe
they could decrease the number of carts needed by sometimes sending off a cart
that was less full, and in this way allow the next few carts to be better
packed. However, note that your family is not willing to change the order
that the bins are packed into the carts, since that would require a large loading area and
extra work to rearrange the bins.

\begin{enumerate}[(a)]
    \item (1 point) Give an example set of weights $w_1, w_2, \ldots, w_{10}$ and a maximum
    cart weight $W$ where your family's greedy algorithm does not fill any cart to capacity.

    \item (1 point) If the loading area was large, and you could rearrange the
    bins in any order you wanted before loading them onto the carts, would that
    help your family use fewer carts? Why or why not?

    \item (7 points)
Prove that, for any set of apple bins with specified weights and any maximum
cart weight $W$, the greedy
algorithm currently in use actually minimizes the number of carts that are
needed. Use a proof by exchange argument by induction. See the proof of the
optimality of the interval scheduling greedy algorithm in class and the proof
of the optimality of the advertiser slot greedy algorithm on PrairieLearn as examples,
and consider focusing on the last bin added to each cart.

\item (1 point) What outside resources did you use to help with this problem?
\end{enumerate}

\end{enumerate}
    %%%%%%%%%%%%%%%%%%%%
    \end{document}
