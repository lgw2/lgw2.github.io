% ---------
% --------
%!TEX TS-program = pdflatex
%!TEX encoding = UTF-8 Unicode
%!TEX spellcheck = English (Aspell)

\documentclass[11pt]{article}
\usepackage{jeffe,handout,graphicx}
\usepackage{fourier-orns}
\usepackage{mathtools}
\usepackage[normalem]{ulem} % either use this (simple) or
\usepackage{tikz}

\def\Cdot{\mathbin{\text{\normalfont \textbullet}}}
\def\Sym#1{\texttt{\color{BrickRed}#1}}
\def\BlueSym#1{\textbf{\texttt{\color{RoyalBlue}#1}}}
\allowdisplaybreaks

\begin{document}

\begin{center}
\Large\textbf{CSCI 332, Fall 2025}%
\\
\LARGE\textbf{Homework 1}%
\\[0.5ex]
\large Due Monday, September 1 Anywhere on Earth (6am Tuesday)
\end{center}

\bigskip
\hrule
\bigskip

\subsection*{Submission Requirements}
\begin{itemize}
    \item Type or clearly hand-write your solutions into a PDF format so that they are legible and professional. Submit your PDF on Gradescope. 
    \item Do not submit your first draft. Type or clearly re-write your solutions for your final submission. If your submission is not legible, we will ask you to resubmit.
    \item Use Gradescope to assign problems to the correct page(s) in your solution. If you do not do this correctly, we will ask you to resubmit.
\end{itemize}

\subsection*{Academic Integrity}

Remember, you may access \EMPH{any} resource in preparing your solution to the homework. However, you \EMPH{must}
\begin{itemize}
    \item write your solutions in your own words, and
    \item credit every resource you use (for example: ``Bob Smith helped me on this problem. He took this course at UM in Fall 2020''; ``I found a solution to a problem similar to this one in the lecture notes for a different course, found at this link: www.profzeno.com/agreatclass/lecture10''; ``I asked ChatGPT how to solve  part (c)"; ``I put my solution for part (c) into ChatGPT to check that it was correct and it caught a missing case.'') If you use the provided LaTeX template, you can use the \texttt{sources} environment for this. Ask if you need help!
\end{itemize}

\subsection*{Grading Rubric}




\newpage
%----------------------------------------------------------------------

\headers{CSCI 332}{Homework 1 (due September 1)}{Fall 2025}


\begin{enumerate}
%\parindent 1.5em \itemsep 4ex plus 0.5fil

%----------------------------------------------------------------------

\item 

As an avid pickler, you have volunteered to organize pickleball games for the UM Pickleball club.
In order to assign teams, you wonder whether you could use a stable matching approach.
You ask the $n$ players in the club to rank the remaining $n-1$ players to be their teammate for
an upcoming tournament. 

For example, for $n=4$ players, you might have an instance like: 

\begin{center}
\begin{tabular}{ c c c }
    Alfred: Belinda, Carmen, Daniel \\
    Belinda: Carmen, Alfred, Daniel \\
    Carmen: Alfred, Belinda, Daniel\\
    Daniel: Alfred, Belinda, Carmen
\end{tabular}
\end{center}

\begin{enumerate}[(a)]
    \item (3 points) Re-work the definition of stable matching from the hospital-medical student
        version of the problem to the pickleball team version of the problem.
        You should describe the input of the problem and the desired output of the problem.
    \item (3 points) Does a stable matching always exist for the pickleball team problem?
    \item (3 points) How do you know?
    \item (1 point) What resources did you use for this problem? (If you only used the textbook, lecture notes, and office hours, you can say "none".)
\end{enumerate}

\end{enumerate}
%%%%%%%%%%%%%%%%%%%%
\end{document}
