% ---------
% --------
%!TEX TS-program = pdflatex
%!TEX encoding = UTF-8 Unicode
%!TEX spellcheck = English (Aspell)

\documentclass[11pt]{article}
\usepackage{jeffe,handout,graphicx}
\usepackage{fourier-orns}
\usepackage{mathtools}
\usepackage[normalem]{ulem} % either use this (simple) or
\usepackage{tikz}

\def\Cdot{\mathbin{\text{\normalfont \textbullet}}}
\def\Sym#1{\texttt{\color{BrickRed}#1}}
\def\BlueSym#1{\textbf{\texttt{\color{RoyalBlue}#1}}}
\allowdisplaybreaks

\begin{document}

\begin{center}
\Large\textbf{CSCI 332, Fall 2025}%
\\
\LARGE\textbf{Homework 5}%
\\[0.5ex]
\large Due Monday, September 29 Anywhere on Earth (6am Tuesday)
\end{center}

\bigskip
\hrule
\bigskip

\subsection*{Submission Requirements}
\begin{itemize}
    \item Type or clearly hand-write your solutions into a PDF format so that they are legible and professional. Submit your PDF on Gradescope. 
    \item Do not submit your first draft. Type or clearly re-write your solutions for your final submission. If your submission is not legible, we will ask you to resubmit.
    \item Use Gradescope to assign problems to the correct page(s) in your solution. If you do not do this correctly, we will ask you to resubmit.
\end{itemize}

\subsection*{Academic Integrity}

Remember, you may access \EMPH{any} resource in preparing your solution to the homework. However, you \EMPH{must}
\begin{itemize}
    \item write your solutions in your own words, and
    \item credit every resource you use (for example: ``Bob Smith helped me on
    this problem. He took this course at UM in Fall 2020''; ``I found a solution
    to a problem similar to this one in the lecture notes for a different
    course, found at this link: www.profzeno.com/agreatclass/lecture10''; ``I
    asked ChatGPT how to solve  part (c)"; ``I put my solution for part (c) into
    ChatGPT to check that it was correct and it caught a missing case.'') If you
    use the provided LaTeX template, you can use the \texttt{sources}
    environment for this. Ask if you need help!
\end{itemize}



\newpage
%----------------------------------------------------------------------

\headers{CSCI 332}{Homework 5 (due September 29)}{Fall 2025}

\begin{enumerate}

\item (1 point) Choose any graph or graph algorithm property we have discussed
in class, on a quiz, in a Prairie Learn or written homework and we have not
proved \emph{via induction} (okay if we have proved some other way!) and state
it formally. (This should probably start with something like ``for all graphs of some type,...'') You will
prove it by induction in problem 2. Here are some examples of graph properties
and algorithms we have discussed:
 \begin{itemize}
    \item BFS visits every vertex reachable from the source.
    \item DFS visit every vertex reachable from the source.
    \item The index of the layers produced by BFS are the shortest path
    distances of nodes in that layer from the source.
\end{itemize}

You should be able to find more! Or if there is something you want to prove
about graphs or graph algorithms but we didn't discuss yet in this course, feel
free to ask and I will likely say yes.


\item (8 points) Prove the property or algorithm you stated in problem 1 by induction. No points if you do not use the
exact boilerplate we covered in class and on the Prairie Learn assignment (universal declaration, induction hypothesis, base and inductive case, and wrap up sentence).
\item (1 point) What resources did you use for this assignment? (If you only used the textbook, lecture notes, and office hours, you can say "none".)
    \end{enumerate}
    %%%%%%%%%%%%%%%%%%%%
    \end{document}
